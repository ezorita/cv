%%%%%%%%%%%%%%%%%%%%%%%%%%%%%%%%%%%%%%%%%
% Original author:
% Adrien Friggeri (adrien@friggeri.net)
% https://github.com/afriggeri/CV
%
% Modified:
% Eduard Valera Zorita
% Bibliography works with biblatex 3.3 and biber 2.4.
% Download biblatex from https://sourceforge.net/projects/biblatex/files/biblatex-3.3/
% Download biber from https://sourceforge.net/projects/biblatex-biber/files/biblatex-biber/2.4/
%
% License:
% CC BY-NC-SA 3.0 (http://creativecommons.org/licenses/by-nc-sa/3.0/)
%
% Important notes:
% This template needs to be compiled with XeLaTeX and the bibliography, if used,
% needs to be compiled with biber rather than bibtex.
%
%%%%%%%%%%%%%%%%%%%%%%%%%%%%%%%%%%%%%%%%%

\documentclass[]{friggeri-cv} % Add 'print' as an option into the
                              % square bracket to remove colors from
                              % this template for printing

\usepackage{fontawesome}

\hyphenation{Mo-le-cu-lar bi-o-lo-gy music}

\addbibresource{bibliography.bib} % Specify the bibliography file to include publications

\begin{document}
\newfontfamily{\FA}{FontAwesome}

\header{eduard}{zorita}{scientist, electrical/electronics \& software
  engineer, bioinformatician} % Your name and current job title/field

%----------------------------------------------------------------------------------------
%	SIDEBAR SECTION
%----------------------------------------------------------------------------------------

\begin{aside} % In the aside, each new line forces a line break
  \section{contact}
  {\FA \faUser}
  Eduard Valera Zorita
  {\FA \faEnvelope}
  \href{mailto:eduardvalera@gmail.com}{eduardvalera@gmail}
  {\FA \faLinkedinSign}
  \href{http://www.linkedin.com/in/eduardvalera}{linkedin/eduardvalera}
  {\FA \faGithubSign}
  \href{http://github.com/ezorita}{github.com/ezorita}
  \section{languages}
  catalan
  spanish
  english
  \section{programming}
  C \& C++
  python, Java
  Matlab, R
  PHP, JS-node, SQL
  HTML, CSS
  shell scripting
  VHDL
  \LaTeX, git
  AGILE
  \section{skills}
  {\bf computer science}
  algorithm design
  machine learning
  parallel computing
  automated testing
  databases
  {\bf electrical eng}
  signal processing
  networks
  microprocessors/FPGA
  PCB design
  microelectronics
  control systems
  {\bf biology}
  bioinformatics
  genomics
  molecular biology
  {\bf math}
  {\bf physics}
  {\bf statistics}
  \section{projects}
  \href{http://github.com/gui11aume/starcode}{starcode {\FA \faExternalLink}}
  \href{http://horari.sunion.cat}{horari sunion {\FA \faExternalLink}}
\end{aside}

%----------------------------------------------------------------------------------------
%	INTERESTS SECTION
%----------------------------------------------------------------------------------------

\section{interests}

\textbf{professional:} computer science, artificial intelligence, genomics, molecular
biology. \\
\textbf{personal:} learning, teaching, photography, music, hiking, literature, sports
and popular culture. \\

%----------------------------------------------------------------------------------------
%	WORK EXPERIENCE SECTION
%----------------------------------------------------------------------------------------

\section{experience}

\subsection{Science \& Engineering}

\begin{entrylist}

%------------------------------------------------

\entry
{2014--Now}
{Centre for Genomic Regulation (CRG)}
{Barcelona}
{%
\emph{Research Scientist/Bioinformatician} \\
Research scientist at the Genome Architecture laboratory
(\href{http://blog.thegrandlocus.com/}{Guillaume Filion}). \\
If all cells in a human body have the same genetic information, how come are
they so different? The short answer is:  they express different sets of genes.
For instance, red blood cells produce haemoglobin and neurons need neuroreceptors.
The goal of our team is to understand the mechanisms used by each cell type to
selectively express and repress their genes.\\
{\bf Research fields}: Computer science, statistics, artificial intelligence,
genomics and molecular biology. \\
{\bf Research projects}: gene regulation, HIV latency, 3D conformation of
chromatin, genome alignment and assembly.\\
\href{http://www.genomearchitecture.com}{{\FA \faExternalLink} Lab website.}
}\\

%------------------------------------------------

\entry
{2013--2014}
{Applied Ocean Systems}
{San Diego, CA}
{%
\emph{Electrical Engineer} \\
Communicating underwater is hard. Transmitting live video streams over underwater
acoustic waves sounds almost impossible. In AOS we worked hard to design the
first underwater wireless device capable of transmitting ultracompressed video
through acoustic broadband OFDM. \\
{\bf Main responsibilities}:
  \begin{itemize}
    \item Reserach on cutting-edge underwater communication technology.
    \item Design of algorithms for acoustic signal synchronization \& Doppler compensation.
  \end{itemize}
}\\

%------------------------------------------------

\entry
{2012--2013}
{AUV Lab @ Massachusetts Institute of Technology}
{Cambridge, MA}
{%
\emph{Research Engineer} \\
Marine biologists and oceanographers need high-tech tools to explore the sea.
Building them is the mission of the Autonomous Underwater Vehicle laboratory.
The mission of our team of three engineers was to design and manufacture unmanned
and remotely-controlled underwater vehicles, used in all sorts of scientific expeditions.\\
{\bf Main responsibilities}:
  \begin{itemize}
    \item Onboard hardware and software design for autonomous
      underwater vehicles.
    \item Research on communication technologies for underwater
      vehicles.
    \item Development of autopilot and sensor drivers.
  \end{itemize}
\href{http://seagrant.mit.edu}{{\FA\faExternalLink} Department website.}
}

\end{entrylist}
% Allow Page break
\begin{entrylist}

%------------------------------------------------

\entry
{2009--2010}
{Signal Theory \& Communications department @ UPC}
{Barcelona}
{%
\emph{Research assistant} \\
I worked as research assistant under the supervision of Prof.\ Josep
Vidal. Our line of research focused on designing multi-antenna communication
techniques for the 4G wireless standard. \\
{\bf Research topics:} wireless communications, signal processing,
algebra, array processing (MIMO), convex optimization.
}\\

%------------------------------------------------

\entry
{2009--2011}
{Sunion ICC}
{Barcelona}
{%
\emph{Software Developer} \\
I developed a platform to create and publish dynamic weekly class schedules.
The project was developed for a high-school based in Barcelona. The platform
consists of a class schedule editor, database server, screen visualization
system and mobile/web app. \\
\href{http://horari.sunion.net}{{\FA\faExternalLink} Web/mobile app.}
}
%------------------------------------------------


\end{entrylist}

\subsection{Teaching}

\begin{entrylist}

\entry
{2015--2016}
{School of Molecular and Theoretical Biology}
{Pushchino, Russia/Barcelona}
{\emph{Faculty}\\
I participated as faculty in two editions of the School of Theoretical
and Molecular biology held in Pushchino, Russia and Barcelona on
August 2015/2016, respectively.\\
{\em Projects taught:}\\
{\bf Laboratory of DNA manipulation}. The students learned how to clone specific DNA
sequences from major Eukaryote species in an actual molecular biology laboratory. \\
{\bf Laboratory of Yeast transformation}. The students learned how to make genetically
modified yeast in an actual molecular biology laboratory. \\
{\em Skills taught:}
\begin{itemize}
  \item Molecular biology.
  \item Basic microbiology.
  \item Basic bioinformatics.
  \item DNA cloning.
  \item Yeast culture and growth.
\end{itemize}
}\\


%------------------------------------------------

\entry
{2015}
{Bioinformatics, Laboratory Course}
{Universitat Pompeu Fabra, Barcelona}
{\emph{Human Biology degree} \\
I taught a laboratory project on basic bioinformatics. The students
used basic bioinformatic tools and programming knowledge to
identify selenoproteins through sequence analysis and protein
structure prediction.} \\

%------------------------------------------------

\entry
{2010--2011}
{Physics, Course I}
{Universitat Politecnica de Catalunya, Barcelona}
{\emph{Electrical Engineering degree} \\
  I taught supplementary classes for an undergraduate course on Physics. \\
  Teaching evaluation awards:
  \begin{itemize}
    \item Most attended course.
    \item Best attendee performance.
  \end{itemize}
}


\end{entrylist}

% Page Break
\newpage
%----------------------------------------------------------------------------------------
%	EDUCATION SECTION
%----------------------------------------------------------------------------------------

\section{education}

\begin{entrylist}

%------------------------------------------------

\entry
{2012--2013}
{Research Engineer}
{Massachusetts Institute of Technology, Cambridge (MA)}
{I was enrolled at the Department of Mechanical Engineering,
  where I conducted research on autonomous underwater vehicles and
  underwater wireless communications.}

%------------------------------------------------

\entry
{2011--2012}
{MSc Thesis in Electrical Engineering}
{Northeastern University, Boston (MA)}
{Master's Thesis at the Digital Signal Processing laboratory under the
  supervision of \href{http://millitsa.coe.neu.edu/}{Milica
    Stojanovic}: Underwater communications.\\
  The study was also supported by the Massachusetts Institute of
  Technology.\\
  \href{http://seagrant.mit.edu/publications/MITSG_12-15.pdf}{{\FA \faExternalLink} Thesis.}
}
%------------------------------------------------

\entry
{2011--2014}
{MSc Electronics Engineering}
{Universitat Politecnica de Catalunya, Barcelona}
{Main subjects: semiconductor physics, electronic and photonic
  devices, microelectronic layer design, FPGA/microcontroller systems
  design, feedback control circuits.\\
  \href{http://upcommons.upc.edu/handle/2117/97991}{{\FA \faExternalLink} Thesis.}
}

%------------------------------------------------

\entry
{2006--2011}
{BSc \& MSc Electrical Engineering}
{Universitat Politecnica de Catalunya, Barcelona}
{Main subjects: math, statistics/probability, physics, circuit theory,
  electronics, programming, computer architecture, communication
  theory, information science, antennas, networks, optical
  communications, advanced signal processing, machine learning,
  criptography and digital security, quantum computing.
}

%------------------------------------------------

\entry
{2008--2011}
{BSc Physics}
{Universitat de Barcelona}
{Degree not completed, 3 years out of 4 finished.\\
Main subjects: math, mechanics, electromagnetism, thermodynamics,
optics, quantum physics, relativity, particle physics.
}

%------------------------------------------------

\end{entrylist}


%----------------------------------------------------------------------------------------
%	VOLUNTEERING SECTION
%----------------------------------------------------------------------------------------

\section{volunteering}

\begin{entrylist}

%------------------------------------------------

\entry
{2013--2015}
{ALS palliative care}
{Fundacio Miquel Valls, Barcelona}
{I provided weekly palliative care to patients with Amyotrophic
  Lateral Sclerosis, a fatal motor neuron disease.}

%------------------------------------------------

\entry
{2011}
{Education through sport}
{Uvikiuta Organization, Tanzania, Africa}
{We used sports and games to assist education and cultural exchange with
  primary school kids in Dar es Salaam, Tanzania.}


\end{entrylist}

%----------------------------------------------------------------------------------------
%	PUBLICATIONS SECTION
%----------------------------------------------------------------------------------------

\section{publications}

% Print all articles from the bibliography
\printbibsection{article}{Journal articles}

% Conference papers
\begin{refsection} 
\nocite{*}
\printbibliography[sorting=chronological, type=inproceedings,
  title={Conference papers}, heading=subbibnumbered]
\end{refsection}

%% \begin{refsection} % This is a custom heading for those references marked as "inproceedings" and containing "keyword=france"
%% \nocite{*}
%% \printbibliography[sorting=chronological, type=inproceedings,
%% title={local peer-reviewed conferences/proceedings},
%% keyword={france}, heading=subbibnumbered]
%% The opposite would be
%% \printbibliography[sorting=chronological, type=inproceedings,
%% title={international conference papers}, notkeyword={france}, heading=subbibnumbered]
%% \end{refsection}

% Print all phdthesis entries from the bibliography
\printbibsection{phdthesis}{MSc theses} 

%----------------------------------------------------------------------------------------

\end{document}
