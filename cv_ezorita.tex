%%%%%%%%%%%%%%%%%%%%%%%%%%%%%%%%%%%%%%%%%
% Original author:
% Adrien Friggeri (adrien@friggeri.net)
% https://github.com/afriggeri/CV
%
% Modified:
% Eduard Valera Zorita
% Bibliography works with bibtex 3.3+ and biber.
%
% License:
% CC BY-NC-SA 3.0 (http://creativecommons.org/licenses/by-nc-sa/3.0/)
%
% Important notes:
% This template needs to be compiled with XeLaTeX and the bibliography, if used,
% needs to be compiled with biber rather than bibtex.
%
%%%%%%%%%%%%%%%%%%%%%%%%%%%%%%%%%%%%%%%%%

\documentclass[]{friggeri-cv} % Add 'print' as an option into the
                              % square bracket to remove colors from
                              % this template for printing

\usepackage{fontawesome}

\addbibresource{bibliography.bib} % Specify the bibliography file to include publications

\begin{document}
\newfontfamily{\FA}{FontAwesome}

\header{eduard}{zorita}{scientist, electrical/electronics \& software
  engineer, bioinformatician} % Your name and current job title/field

%----------------------------------------------------------------------------------------
%	SIDEBAR SECTION
%----------------------------------------------------------------------------------------

\begin{aside} % In the aside, each new line forces a line break
  \section{contact}
  {\FA \faUser}
  Eduard Valera Zorita
  {\FA \faEnvelope}
  \href{mailto:eduardvalera@gmail.com}{eduardvalera@gmail}
  {\FA \faLinkedinSign}
  \href{http://www.linkedin.com/in/eduardvalera}{linkedin/eduardvalera}
  {\FA \faGithubSign}
  \href{http://github.com/ezorita}{github.com/ezorita}
  \section{languages}
  catalan
  spanish
  english
  \section{programming}
  C \& C++
  python, Java
  Matlab, R
  PHP, django, SQL
  HTML, CSS
  shell scripting
  VHDL
  \LaTeX, git
  \section{skills}
  computer science
  algorithm design
  machine learning
  parallel computing
  automated testing
  databases

  electrical eng
  signal processing
  networks
  microprocessors/FPGA
  PCB design
  firmware programming
  control systems

  biology
  bioinformatics
  genomics
  molecular biology

  math
  physics
  statistics
  \section{projects}
  \href{http://github.com/gui11aume/starcode}{starcode}
  \href{http://horari.sunion.cat}{horari sunion}
\end{aside}

%----------------------------------------------------------------------------------------
%	INTERESTS SECTION
%----------------------------------------------------------------------------------------

\section{interests}

\textbf{professional:} science, creativity, algorithms, big data,
biology \textbf{personal:} learning, teaching, music, hiking,
photography, circus arts, literature, theater, sports and popular culture.

%----------------------------------------------------------------------------------------
%	EDUCATION SECTION
%----------------------------------------------------------------------------------------

\section{education}

\begin{entrylist}

%------------------------------------------------

\entry
{2012--2013}
{Research Scholar}
{Massachusetts Institute of Technology, Cambridge (MA)}
{I was enrolled at the Department of Mechanical Engineering,
  where I conducted research on autonomous underwater vehicles and
  underwater wireless communications.}

%------------------------------------------------

\entry
{2011--2012}
{MSc Thesis in Electrical Engineering}
{Northeastern University, Boston (MA)}
{Master's Thesis at the Digital Signal Processing laboratory under the
  supervision of \href{http://millitsa.coe.neu.edu/}{Milica
    Stojanovic}: Underwater communications.\\
  The study was also supported by the Massachusetts Institute of
  Technology.\\
  \href{http://seagrant.mit.edu/publications/MITSG_12-15.pdf}{{\FA \faExternalLink} Thesis.}
}
%------------------------------------------------

\entry
{2011--2014}
{MSc Electronics Engineering}
{Universitat Politecnica de Catalunya, Barcelona}
{Main subjects: semiconductor physics, electronic and photonic
  devices, microelectronic layer design, FPGA/microcontroller systems
  design, feedback control circuits.\\
  \href{http://upcommons.upc.edu/handle/2117/97991}{{\FA \faExternalLink} Thesis.}
}

%------------------------------------------------

\entry
{2006--2011}
{BSc \& MSc Electrical Engineering}
{Universitat Politecnica de Catalunya, Barcelona}
{Main subjects: math, statistics/probability, physics, circuit theory,
  electronics, programming, computer architecture, communication
  theory, information science, antennas, networks, optical
  communications, advanced signal processing, machine learning,
  criptography and digital security, quantum computing.
}

%------------------------------------------------

\entry
{2008--2011}
{BSc Physics}
{Universitat de Barcelona}
{Degree not completed, 3 years out of 4 finished.\\
Main subjects: math, mechanics, electromagnetism, thermodynamics,
optics, quantum physics, relativity, particle physics.
}

%------------------------------------------------

\end{entrylist}

%----------------------------------------------------------------------------------------
%	WORK EXPERIENCE SECTION
%----------------------------------------------------------------------------------------

\section{experience}

\subsection{Science \& Engineering}

\begin{entrylist}

%------------------------------------------------

\entry
{2014--Now}
{Centre for Genomic Regulation (CRG)}
{Barcelona}
{%
\emph{Research Scientist/Bioinformatician} \\
Research scientist at the Genome Architecture laboratory
(\href{http://blog.thegrandlocus.com/}{Guillaume Filion}). At the lab
I work on algorithmically complex problems derived from experimental
data analyses, such as sequence clustering, sequence alignment and
pattern matching. I also analyze bioinformatic data and
contribute to the design of molecular biology experiments.\\
{\bf Research fields}: computer science, computational biology,
genomics and molecular biology. \\
{\bf Research projects}: HIV latency, genome alignment and assembly, DNA
sequence clustering.\\
\href{http://www.genomearchitecture.com}{{\FA \faExternalLink} Lab website.}
}

%------------------------------------------------

% Page break
\end{entrylist}
\begin{entrylist}

\entry
{2013--2014}
{Applied Ocean Systems}
{San Diego, CA}
{%
\emph{Electrical Engineer} \\
Remotely worked on a small company determined to launch the first
underwater wireless communication device capable of transmitting live
video stream. \\
{\bf Main responsibilities}:
  \begin{itemize}
    \item Design of cutting-edge communication/modulation technology.
    \item Algorithms for signal synchronization \& Doppler compensation.
  \end{itemize}
}

%------------------------------------------------

\entry
{2012--2013}
{AUV Lab @ Massachusetts Institute of Technology}
{Cambridge, MA}
{%
\emph{Research Engineer} \\
I worked as communications and electrical engineer at the Autonomous
Underwater Vehicles laboratory. Our research lab was comprised of
three engineers and our goal was to design and manufacture unmanned
and remotely-controlled underwater vehicles for diverse scientific
missions.
{\bf Main responsibilities}:
  \begin{itemize}
    \item Onboard hardware and software design for autonomous
      underwater vehicles.
    \item Design of communication technologies for underwater
      vehicles.
    \item Design of autopilot and sensor drivers for an autonomous vehicle.
  \end{itemize}
\href{http://seagrant.mit.edu}{{\FA\faExternalLink} Department website.}
}

%------------------------------------------------

\entry
{2009--2010}
{Signal Theory \& Communications department @ UPC}
{Barcelona}
{%
\emph{Research assistant} \\
I worked as research assistant under the supervision of Prof.\ Josep
Vidal. Our line of research focused on communication techniques for the 4G
wireless standard. \\
{\bf Research topics:} wireless communications, signal processing,
algebra, antenna array processing (MIMO).
}


%------------------------------------------------

\entry
{2009--2011}
{Sunion ICC}
{Barcelona}
{%
\emph{Software Developer} \\
I developed a platform to provide digital support to a dynamic class
schedule system. The project was specially developed for a
secondary school based in Barcelona. The platform has a desktop editor
software, database server, screen visualization system and mobile/web
app. \\
\href{http://horari.sunion.net}{{\FA\faExternalLink} Web/mobile app.}
}
%------------------------------------------------


\end{entrylist}

\subsection{Teaching}

\begin{entrylist}

\entry
{2015--2016}
{School of Molecular and Theoretical Biology}
{Pushchino, Russia/Barcelona}
{\emph{Faculty}\\
I participated as faculty in two editions of the School of Theoretical
and Molecular biology held in Pushchino, Russia and Barcelona on
August 2015/2016, respectively.\\
Projects taught:\\
{\bf Laboratory of DNA manipulation}. The students learned how to clone specific DNA
sequences from major Eukaryote species in a real laboratory
project. They further analyzed and presented the obtained results
following the scientific method. {\em Skills taught}: basic microbiology,
molecular biology, DNA manipulation and basic bioinformatics. \\
{\bf Laboratory of Yeast transformation}. The students learned how to make genetically
modified yeast in a real laboratory project. They further analyzed and
presented the obtained results following the scientific method. {\em Skills
thaught}: Yeast culture and growth, molecular biology.
}

%------------------------------------------------

\entry
{2015}
{Bioinformatics, Laboratory Course}
{Universitat Pompeu Fabra, Barcelona}
{\emph{Human Biology degree} \\
I taught a laboratory project on basic bioinformatics. The students
had to use basic bioinformatic tools and programming knowledge to
identify selenoproteins through sequence analysis and protein
structure prediction.} 

%------------------------------------------------

% Page break
\end{entrylist}
\begin{entrylist}

\entry
{2010--2011}
{Physics, Course I}
{Universitat Politecnica de Catalunya, Barcelona}
{\emph{Electrical Engineering degree} \\
I taught an undergraduate course on Physics. Teaching evaluation
awards: Most attended course and best attendee performance.
}


\end{entrylist}

%----------------------------------------------------------------------------------------
%	VOLUNTEERING SECTION
%----------------------------------------------------------------------------------------

\section{volunteering}

\begin{entrylist}

%------------------------------------------------

\entry
{2013--2015}
{ALS palliative care}
{Fundacio Miquel Valls, Barcelona}
{I provided weekly palliative care to patients with Amyotrophic
  Lateral Sclerosis, a fatal motor neuron disease.}

%------------------------------------------------

\entry
{2011}
{Education through sport}
{Uvikiuta Organization, Tanzania, Africa}
{We used sports and games to assist education and cultural exchange with
  primary school kids in Dar es Salaam, Tanzania.}


\end{entrylist}

%----------------------------------------------------------------------------------------
%	PUBLICATIONS SECTION
%----------------------------------------------------------------------------------------

\section{publications}

% Print all articles from the bibliography
\printbibsection{article}{Journal articles}

% Conference papers
\begin{refsection} 
\nocite{*}
\printbibliography[sorting=chronological, type=inproceedings,
  title={Conference papers}, heading=subbibnumbered]
\end{refsection}

%% \begin{refsection} % This is a custom heading for those references marked as "inproceedings" and containing "keyword=france"
%% \nocite{*}
%% \printbibliography[sorting=chronological, type=inproceedings,
%% title={local peer-reviewed conferences/proceedings},
%% keyword={france}, heading=subbibnumbered]
%% The opposite would be
%% \printbibliography[sorting=chronological, type=inproceedings,
%% title={international conference papers}, notkeyword={france}, heading=subbibnumbered]
%% \end{refsection}

% Print all phdthesis entries from the bibliography
\printbibsection{phdthesis}{MSc theses} 

%----------------------------------------------------------------------------------------

\end{document}
